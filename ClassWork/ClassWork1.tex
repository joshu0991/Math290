\documentclass[11pt]{article}
\usepackage{enumitem}
\usepackage{amsmath}
\usepackage{graphicx}

\begin{document}

Wayne Green

Victoria Haley

Joshua Lilly

Math 290 - Group Assignment I

8/31/2016
\begin{flushleft}

\begin{enumerate}[widest={5.2}]
\item[2a.] \begin{center}
\begin{tabular}{| l | l | l | l | l | l | l | l |}
  \hline
  P & Q & R & (P $\vee$ R) & (P $\wedge$ Q) & (P $\wedge$ R) & (P $\wedge$ (Q $\vee$ R)) & (P $\wedge$ Q) $\vee$  (P $\wedge$ R) \\ \hline
  F & F & F & F & F & F & F & F\\ \hline
  F & F & T & T & F & F & F & F\\ \hline
  F & T & F & T & F & F & F & F\\ \hline
  F & T & T & T & F & F & F & F\\ \hline
  T & F & F & F & F & F & F & F\\ \hline
  T & F & T & T & F & T & T & T\\ \hline
  T & T & F & T & T & F & T & T\\ \hline
  T & T & T & T & T & T & T & T\\ \hline
  \end{tabular}
\end{center}

\item[2b.]
Not possible for this problem since the problem statement is the definition of the distributive property of and over or
  
\item[2c.]
In order to show that P $\wedge$ (Q $\vee$ R) is equivalent to 

(P $\wedge$ Q) $\vee$ (P $\wedge$ R) we must show two things.
 
\vspace{5mm} %5mm vertical space
(1) P $\wedge$ (Q $\vee$ R) then 
(P $\wedge$ Q) $\vee$ (P $\wedge$ R)\\
(2) (P $\wedge$ Q) $\vee$ (P $\wedge$ R) then P $\wedge$ (Q $\vee$ R)
\bigskip

(1) Suppose P $\wedge$ (Q $\vee$ R) is true. In this case we must also show that\\ (P $\wedge$ Q) $\vee$ (P $\wedge$ R) is true. For (P $\wedge$ Q) $\vee$ (P $\wedge$ R) to be true either (P $\wedge$ Q) must be true OR (P $\wedge$ R) must be true. Since 
P $\wedge$ (Q $\vee$ R) is true, P must be true and either Q OR R must be true. Since P must be true and either Q OR R must be true, then it follows that at least 
(P $\wedge$ Q) is true OR (P $\wedge$ R) is true.  
\smallskip

(2) assume (P $\wedge$ R) is true. If (P $\wedge$ R) is true then it must be the case that P AND R are both true. Since P AND R are both true, since P AND R must be true, then (Q $\vee$ R) must be true since R is true. and the statement P $\wedge$ (Q $\vee$ R) must also be true since P is also true. Therefore, P $\wedge$ (Q $\vee$ R) and (P $\wedge$ Q) $\vee$ (P $\wedge$ R) must be equivalent.
 
\item[4a.] 
 \begin{center}
 \scalebox{0.91}{
\begin{tabular}{| l | l | l | l | l | l | l | l |}
  \hline
  P & Q & R & (Q $\wedge$ R) & (P $\implies$ R) & (P $\implies$ Q) & (P $\implies$ (Q $\wedge$ R)) & (P $\implies$ Q) $\wedge$  (P $\implies$ R) \\ \hline
  F & F & F & F & F & F & F & F\\ \hline
  F & F & T & T & F & F & F & F\\ \hline
  F & T & F & T & F & F & F & F\\ \hline
  F & T & T & T & F & F & F & F\\ \hline
  T & F & F & F & F & F & F & F\\ \hline
  T & F & T & T & F & T & T & T\\ \hline
  T & T & F & T & T & F & T & T\\ \hline
  T & T & T & T & T & T & T & T\\ \hline
  \end{tabular}
  }
\end{center}
\medskip

\item[4b]
 (P $\implies$ (Q $\wedge$ R)) $\equiv$ (Implcation removal)\\$\neg$ P $\vee$ (Q $\wedge$ R) $\equiv$ (Distribute)\\ ($\neg$ P $\vee$ Q) $\wedge$ ($\neg$ P $\vee$ R) $\equiv$ (Introduce implications)\\ (P $\implies$ Q) $\wedge$ (P $\implies$ R)
 
 \item[4c]
In order to show that (P $\implies$ (Q $\wedge$ R)) is equivalent to\\ (P $\implies$ Q) $\wedge$ (P $\implies$ R) we must show two things.
\vspace{5mm} %5mm vertical space

(1) (P $\implies$ (Q $\wedge$ R)) then (P $\implies$ Q) $\wedge$ (P $\implies$ R)

(2) (P $\implies$ Q) $\wedge$ (P $\implies$ R) then (P $\implies$ (Q $\wedge$ R))
\bigskip

(1) Assume (P $\implies$ (Q $\wedge$ R)) is true since the statement is true it follows that Q AND R must both be true. This says nothing about P however, since Q AND R both must be true it is not possible for either (P $\implies$ Q) OR (P $\implies$ R) to ever be false since the only way for the implication to be false is if either Q OR R is also false. Since both (P $\implies$ Q) AND (P $\implies$ R) must be true it must be true when (P $\implies$ (Q $\wedge$ R)) is true.
\smallskip

(2) Assume (P $\implies$ Q) $\wedge$ (P $\implies$ R) is true. Which means P, Q AND R can be all false, OR Q AND R can be true. When P, Q, AND R are false. (Q $\wedge$ R) is false, which means since P is false\\ (P $\implies$ (Q $\wedge$ R)) must be true. Which is consistent with\\ (P $\implies$ Q) $\wedge$ (P $\implies$ R). When Q AND R are true, it is not possible for (P $\implies$ Q) $\wedge$ (P $\implies$ R) to ever be false since the implication can only evaluate to false when either Q OR R are false. Additionally, since Q AND R are both true, (Q $\wedge$ R) must also be true, forcing (P $\implies$ (Q $\wedge$ R)) to also always be true. Therefore,\\ (P $\implies$ (Q $\wedge$ R)) must be equivalent to (P $\implies$ Q) $\wedge$ (P $\implies$ R)
\end{enumerate}
 
 
\end{flushleft}
\end{document}